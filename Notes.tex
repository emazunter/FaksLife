\documentclass[a4paper]{article}
\usepackage{hyperref}
\begin{document}
\title{Notes}
\author{Ema in Lan}
\date{}
\maketitle

\section{Storyline}
\subsection*{Vsakdanje življenje}
Si študent na delovni dan in moraš narediti različne naloge, ki povečajo tvoje znanje oziroma tvojo samozavest. Na voljo imaš različne prostore, kjer se lahko učiš, delaš izpite ali pa se družiš s prijatelji. Vsak dan se začne zjutraj in konča zvečer. Na voljo imaš različne aktivnosti, ki vplivajo na tvoje metre (ankoznost, pozornost in samozavest). Meter za anksioznost na primer določa, kakšno možnost imaš, da dobiš panični napad. Pred koncem dneva se lahko odločiš, ali boš šel na zabavo, se celo noč učil ali pa šel spat. Če izbereš celonočno učenje ali zabavo, imaš zjutraj minigame/čimhitreje klikni zaslon, ker si zaspal in imaš posledično visoko anksioznost že zjutraj. Kava ti izboljša pozornost in samozavest, a poslabša anksioznost. Učenje ti izboljša samozavest, a poslabša pozornost in anksioznost.

\subsection*{Dan izpita}
Pisni izpiti so enaki a vse igralce. Imaš pravilne in napačne odgovore, težavnost je odvisna od stanja tvojih metrov. Če mas na primer meter za pozornost prenizek, se ti izpit zdi težji, kot je v resnici. Ustni izpiti so bolj odvisni od tvojih metrov. Če imaš visoko anksioznost, boš imel težave pri ustnem izpitu. Na voljo imaš tudi različne predmete, ki jih lahko izbereš, preden opravljaš izpit. Na voljo so tudi različni profesorji, ki imajo različne učne sloge in težavnosti. Če uspešno opraviš izpit, dobiš točke in napreduješ v igri. Če izpit padeš, se igrica konča. Tvoja naloga je, da uspešno opraviš vse izpite in postaneš najboljši študent na fakulteti.

\section{Prostori}
\begin{itemize}
    \item MAFIJA\: malica, kava, učenje, druženje
    \item KABINET\: ustni izpiti
    \item 2. ŠTUK\: učenje, wc, pouk in pisni izpiti 
    \item KNJIŽNICA (pogojno): učenje in vračanje knjig (zamudnina)
    \item SOBA\: učenje, druženje, kava, spanje, jutranji minigame
\end{itemize}

\section{Playable igralci}
\begin{itemize}
    \item first (Ema)
    \item second (Lan)
\end{itemize}

\section{NPC-ji}
\begin{enumerate}
    \item Profesorji: \begin{itemize}
        \item Pavel Peteršič (stop)
        \item Tomaž Različnik (fiz)
    \end{itemize}
    \item bariste
    \item knjižničarka
    \item študenti: \begin{itemize}
        \item Bor
        \item Matija
    \end{itemize}
\end{enumerate}

\section{Izpiti}
\begin{enumerate}
    \item Pisni izpiti: \begin{itemize}
        \item kratki, zelo odvisni od metra za pozornost in anksioznost
    \end{itemize}

    \item Ustni izpiti: \begin{itemize}
        \item STOP\: \ldots
        \item Fizika: \begin{itemize}
            \item ponesreči zaokrožiš $\Pi$ na 5 namesto na 3
            \item imaš možnost izpustiti neko spremenljivko, ker se ti ne zdi\\
            pomembna in 50 \% možnost, da je bila to pravilna odločitev
        \end{itemize}
    \end{itemize}
\end{enumerate}

\section{Interaktivne stvari}
\begin{itemize}
    \item kupiš kavo
    \item vrneš knjigo
    \item pišeš izpit
    \item opraviš ustni izpit
    \item porabiš bon za tortilijo
    \item učiš se v knjižnici ali v kavarni
    \item greš na čik (zniža anksioznost)
    \item poslušaš glasbo med predavanjem (zniža pozornost in anksioznost)
\end{itemize}

\section{Igralci in NPC-ji podrobneje}
\subsection*{Bor}
\begin{itemize}
    \item add unskippable yap scene, if you have a phone, you can distract him with anime brainrot
\end{itemize}

\section{Manjše nevšečnosti}
\begin{itemize}
    \item spotakneš se
    \item nekdo te zbije z vrati wcja
    \item v wcju ni brisačk in moraš na ustni izpit z mokrimi rokami
    \item profesor zavleče uro in zamudiš na izpit
    \item zaspiš in zamudiš pouk
    \item spotakneš se
    \item ne moreš najti lista s formulami
    \item pišeš izpit iz prog in nimaš naloženga vs code na šolskem računalniku
    \item pri fiziki moraš uporabljati orodje, ki ga še nisi spoznal (Izračunaj rezultanto sil na klancu, pri čemer si pomagaj z Riemannovo hipotezo)
\end{itemize}

\section{Extra}
\begin{itemize}
    \item več playable igralcev with quirks
    \item več letnikov + več izpitov
    \item višji kot imas meter za anksioznost, večji je volumen glasbe
    \item moraš it bruhat pred izpitom (Ema)
\end{itemize}

\section{Soundtracks}
\begin{itemize}
    \item \textbf{copyrighted}: \begin{itemize}
        \item Your best nightmare: Undertale \\
        \textsl{(anxiety)}
        \item On the Movements of the Earth OST \- Main Theme (HQ Cover): Kensuke Ushio Orb \\
        \textsl{(exams)}
        \item Battle! Champion Cynthia: `Pokémon Diamond`\\
        \textsl{(exams \#2)}
    \end{itemize}
    
    \item \textbf{non-copyrighted}: \begin{itemize}
        \item GO\: Sam Day (ncs) \\
        \textsl{(morning)}
        \item first-main: Lan
        \textsl{still very bad, needs work}
    \end{itemize}
    \item sound effects: pixabay in freesound
\end{itemize}

\section{Art}
dokler še ni narejen art, najdta free assets za engine (assetstore)
\begin{itemize}
    \item pixelart (piskel \- software, lospec \- colorschemes)
    \item non-pixelart (any digital drawing app, recimo: procreate, gimp, krita)
\end{itemize}

\section{Development}
\begin{itemize}
    \item \textbf{Način dela}: \begin{itemize}
        \item knjižnica sauron: \href{https://docs.rs/sauron/latest/sauron/index.html}{Dokumentacija}
    \end{itemize}
\end{itemize}


\end{document}


