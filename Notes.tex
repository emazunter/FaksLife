\documentclass[a4paper]{article}
\begin{document}
\title{Notes}
\author{Ema in Lan}
\date{}
\maketitle

\section{Storyline}
\subsection*{Day-to-day}
Si študent na solski dan in moraš narediti tasks ki povečajo tvoje znanje oz tvoj confidence.
Odlocitve, ki jih delas, vplivajo na tvoj anxiety, attention in confidence meter: anxiety meter doloca kako likely si da dobiš panic attack. Pred koncem dneva se lahko odločiš, ali boš šel na zabavo, naredil allnighter ali pa šel spat. Vse to vpliva na metre. Ce izberes allnighter ali zabavo imas zjutraj minigame/click the screen as fast as you can, ker si zaspal in imas posledicno visok anxiety ze zjutraj. Kava ti izboljša attention in confidence ampak poslabša anxiety

\subsection*{Dan izpita}
Pisni izpiti so enaki a vse characterje. Imas pravilne in napacne odgovore, težavnost je odvisna od stanja tvojih metrov.
Če mas attention meter prenizek, se zmedeš med izpitom.
Če izpit padeš, se igrica konča.

\section{Prostori}
\begin{itemize}
    \item MAFIJA: eat, coffee, maybe study if no library
    \item KABINET: ustni zagovori
    \item 2. ŠTUK: pouk in pisni izpiti
    \item LIBRARY(pogojno): study in vracanje knjige (zamudnina)
\end{itemize}

\section{Playable characters}
\begin{itemize}
    \item female (Ema)
    \item male (Lan)
\end{itemize}

\section{NPC-ji}
\begin{enumerate}
    \item Profesorji: \begin{itemize}
        \item Pavel Peteršič (stop)
        \item Tomaž Različnik (fiz)
    \end{itemize}
    \item Baristas
    \item Knjiznicarka
    \item Studenti: \begin{itemize}
        \item Bor
        \item Matija
    \end{itemize}
\end{enumerate}

\section{Izpiti}
\begin{enumerate}
    \item Pisni izpiti: \begin{itemize}
        \item kratki, zelo odvisni od attention bar in anxiety bar
    \end{itemize}

    \item Ustni izpiti: \begin{itemize}
        \item STOP: \ldots
        \item Fizika: \begin{itemize}
            \item ponesreci zaokrozis $\Pi$ na 5 namesto na 3
            \item imas moznost izpistiti neko spremenljivko, ker se ti ne zdi pomembna in 50 \% moznost, da je bila to pravilna odlocitev
        \end{itemize}
    \end{itemize}
\end{enumerate}

\section{Interactive stvari}
\begin{itemize}
    \item kupiš kavo
    \item vrneš knjigo
    \item pišeš izpit
    \item opraviš ustni izpit
    \item porabis bon za tortilijo
    \item učiš se v knjižnici ali v kavarni
    \item 
\end{itemize}

\section{Characters indepth}
\subsection*{Bor}
\

\section{Minor inconveniences}
\begin{itemize}
    \item spotaknes se
    \item nekdo te zbije z vrati wcja
    \item v wcju ni brisack in mors na ustnega z mokrimi rokami
    \item profesor zavlece uro in zamudis na izpit
    \item zaspis in zamudis pouk
    \item pises izpit iz prog in nimas nalogenga vs code na solskem racunalniku
    \item pri fiziki moras uporabljati rodje, ki ga se nisi spoznal (Izracunaj rezultanto sil na klancu, pri čemer si pomagaj z Riemannovo hipotezo)
\end{itemize}

\section{Extra}
\begin{itemize}
    \item dobis panic attack na izpitu
    \item več playable characters with quirks
    \item več letnikov + več izpitov
    \item visji kot imas anxiety meter, vidji je volumen muske
    \item mors it bruhat pred izpitom (Ema)
\end{itemize}

\section{Soundtracks}
Make your own music: Beepbox
\begin{itemize}
    \item \textbf{copyrighted}: \begin{itemize}
        \item Your best nightmare: Undertale \\
        \textsl{(anxiety)}
        \item On the Movements of the Earth OST - Main Theme (HQ Cover): Kensuke Ushio Orb \\
        \textsl{(exams)}
        \item Battle! Champion Cynthia: "Pokémon Diamond"\\
        \textsl{(exams \#2)}
    \end{itemize}
    
    \item \textbf{non-copyrighted}: \begin{itemize}
        \item GO: Sam Day (ncs) \\
        \textsl{(morning)}
    \end{itemize}
    \item sound effects: pixabay in freesound
\end{itemize}

\section{Art}
dokler se ni narejen art, najdta free assets za engine (assetstore)
\begin{itemize}
    \item pixelart (piskel - software, lospec - colorschemes)
    \item non-pixelart (any digital drawing app, recimo: procreate, gimp, krita)
\end{itemize}

\section{Development}
\begin{itemize}
    \item \textbf{Način dela}: \begin{itemize}
        \item knjižnica SDL2 (delaš od začetka vse, vse je bl logicno in ko znas osnovo se da lepo gradit na njej, bor pomaga) \ref{https://crates.io/crates/sdl1_2-rs}{naslov crate}
        \item Game engines: Bevy (nova stvar zato lahko da stvari se ne delajo, easier)
    \end{itemize}
\end{itemize}


\end{document}


